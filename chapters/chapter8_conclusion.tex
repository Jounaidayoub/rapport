% Chapter 8: Conclusion
\chapter{Conclusion}

\section{Summary of Achievements}
This report presents the work completed during my internship witn LEYTON COGNITX, where I was tasked with designing and implementing a real-time financial data surveillance system. The main accomplishments of this project include:
\begin{itemize}
    \item Building a scalable, event-driven architecture based on microservices.
    \item Integrating Apache Kafka to handle high-throughput data streaming.
    \item Setting up a robust anomaly detection system using Celery and Redis.
    \item Using Elasticsearch for fast storage and retrieval of anomaly data.
    \item Developing a FastAPI-based API for querying and reporting.
    \item Containerizing all services with Docker and Docker Compose to simplify deployment.
\end{itemize}
The system is able to ingest, process, and analyze financial data in real time, detect anomalies, and provide actionable insights.

\section{Lessons Learned and Future Perspectives}

This internship with LEYTON COGNITX was a great opportunity to work on a complex distributed system and deepen my understanding of modern backend architectures, moving well beyond traditional monolithic or CRUD-based applications.

\subsection{Scalability and Complexity of Distributed Systems}
Working with microservices and distributed systems showed me how powerful and flexible they can be. Each service can be developed, deployed, and scaled independently, which is a big advantage for handling different loads in a financial context. However, I also saw firsthand that this flexibility comes with added complexity—especially when it comes to communication between services, keeping data consistent, and managing everything in production. I learned that while distributed systems are great for scaling and resilience, they require careful design and strong monitoring to keep things running smoothly. The balance between simplicity and scalability was very clear throughout the project.

\subsection{Exploring New Technologies}
This project gave me the chance to get hands-on experience with several new technologies:
\begin{itemize}
    \item \textbf{Apache Kafka}: Before this internship, I had never heard of or used Kafka. Discovering how fast, reliable, and scalable it is was a real eye-opener. I learned how major companies use Kafka for real-time data streaming, and out of curiosity about what makes it so fast, I discovered some technical details such as its custom network protocol, sequential disk writes, and the zero-copy principle with the \texttt{sendfile} syscall—all of which help explain its high performance. ,This  opened my eyes to a  new lower-level aspects of system development , which I found genuinely interesting.
    \item \textbf{API/Worker Pattern (Celery \& Redis)}: Implementing asynchronous task processing with Celery and Redis showed me how to offload heavy computations from the main API, making the system more responsive and easier to scale. This pattern is essential for building modern, scalable backend services.
    \item \textbf{Elasticsearch}: Working with Elasticsearch gave me a sense of its power as a search and analytics engine, especially for time-series data. Its near real-time indexing and strong query features are a big advantage for applications that need fast data retrieval and aggregation—much more than what traditional relational databases can offer for these use cases.
    \item \textbf{Docker \& Containerization}: Although I’d heard a lot about Docker, this was my first time using it. I learned how Docker makes it easy to set up separate containers and services, each with their own dependencies, without worrying about environment conflicts. I now see how Docker can be useful for reproducible development and deployment in many projects, not just this one.
\end{itemize}
Overall, this internship helped me discover new areas in backend development, especially in data engineering, real-time analytics, and event-driven architectures. It also reinforced the importance of choosing the right tools for building high-performance, resilient systems.
